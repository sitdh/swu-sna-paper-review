\section{\STA}

การเข้าใจโครงสร้าง เพื่อนำข้อมูลมาวิเคราะห์ได้นั้นเป็นใจความสำคัญของหัวข้อนี้
ทั้งนี้เพื่อให้เข้าใจว่าเครือข่ายถูกสร้างขึ้นได้อย่างไร เปรียบเทียบโครงสร้างกันในเครือข่าย 
หรือระหว่างผู้ให้{\SNS}เองก็ตาม หรืออาจสร้างโมเดลของการจับกลุ่มสังคมนี้ก็เป็นได้
ซึ่งการศึกษาที่ผ่านมา พบว่าในเครือข่ายเองมีความสัมพันธ์พิเศษบางอย่าง เช่น 
ระยะทางเฉลี่ยระหว่างโหนดนั้นลดลงเรื่อยๆ 
จำนวนการเชื่อมต่อของโหนดนั้นเป็นไปตามกฎกำลัง (Power law) กล่างคือ
การเชื่อมต่อจะมีมากในโหนดแรกๆ และลดลงไปอย่างรวดเร็วในโหนดที่ไกลออกไป 
ซึ่งภายในหัวข้อนี้มีประเด็นที่น่าสนใจดังนี้

\subsection{คุณลักษณะของโทโพโลจี}
การวิเคราะห์โทโพโลจีของเครือข่ายนั้นเปิดเผยคุณลักษณะที่น่าสนใจของกลุ่มสังคมได้เป็นอย่างดี 
ทั้งวิธีการจับกลุ่มในแนวทางที่ต่างกัน ยิ่งไปกว่านั้น การดึงความสัมพันธ์ออกมาจาก{\OSN} 
นั้นทำได้ง่ายกว่าเครือข่ายแบบออฟไลน์เป็นอย่างมาก ซึ่งในหัวข้อนี้จะมีประเด็นที่น่าสนใจได้แก่

\subsubsection{โครงสร้างของเครือข่ายเปิดเผยอะไรบ้าง} โดยส่วนใหญ่เรามักจะเปรียบเทียบคุณลักษณะ
และโครงสร้างของเครือข่ายใหม่ที่ศึกษา กับเครือข่ายเดิมที่มีข้อมูลอยู่เดิม 
หรือแม้กระทั่งโครงสร้างทางสังคมแบบออฟไลน์ก็ตาม อีกทั้งยังเทียบกับคุณลักษณะเดิมที่ทราบ 
เช่น การเชื่อมต่อระหว่างโหนดในเครือข่ายนี้เป็นไปตามกฎกำลังหรือไม่ หากลุ่มสังคม (Community) ในเครือข่าย
หรือสิ่งที่เครือข่ายนี้กำลังสนใจ เป็นต้น

\subsubsection{ในเครือข่ายหนึ่งประกอบด้วยหลายเครือข่าย} เรื่องที่น่าสนใจอย่างหนึ่งคือ
ภายในเครือข่ายนั้นเอง จะประกอบไปด้วยลักษณะของเครือข่ายย่อยอีกหลายแบบ เช่น 
เครือข่ายของความสนใจที่มีร่วมกันระหว่างโหนด หรือกลุ่มเพื่อน เป็นต้น 

\subsection{คุณลักษณะของการใช้งานและหน้าที่การใช้งาน}
ในบางกรณีนั้นนักวิจัยจะให้ความสนใจเกี่ยวกับความสามารถในการเข้าใจบริการนั้นๆ 
รวมถึงสิ่งที่จะได้ตอบรับกลับมาเมื่อวิเคราะห์การทำงานในเครือข่ายนี้ โดยที่ 

\subsubsection{รูปแบบของเครือข่าย} เป็นลักษณะเด่นที่ต้องให้ความสำคัญเป็นอย่างมาก
เพื่อให้ทราบถึงลักษณะการสร้างขึ้นของเครือข่ายย่อยๆ ภายในเครือข่ายที่สนใจนั้น  

\subsubsection{ลักษณะของผู้ใช้งาน} โดยทั่วไปแล้วจะแบ่งลักษณะผู้ใช้งานออกเป็น 3 ลักษณะด้วยกัน
นั่นคือ \textbf{ผู้กระจายสื่อ (Broadcasters)} ที่มันจะมีผู้ติดตามเป็นจำนวนมาก เช่น
บุคคลที่มีชื่อเสียง หรือเป็นที่รู้จัก \textbf{คนทั่วไป (Acquaintances)} 
เป็นผู้ใช้งานส่วนใหญ่ของระบบที่มักจะมีผู้ติดตามไม่แตกต่างจากจำนวนของคนที่ไปติดตามเท่าไหร่นัก
และสุดท้ายคือ \textbf{ผู้ประสงค์ร้าย (Miscreants)} ที่มีพฤติกรรมติดตามบุคคลอื่นเป็นจำนวนมาก
โดยไม่มีใครติดตามเลย

\subsection{การตรวจสอบสิ่งผิดปรกติและการล่อลวง}
หัวข้อนี้เป็นหัวข้อหนึ่งที่ท้าทายและมีความสำคัญ เนื่องจากเมื่อมีผู้ใช้งานมากยิ่งขึ้น 
โอกาสเกิดความผิดปรกติและการล่อล่วงนั้นย่อมมีมากขึ้นตามไปด้วย ซึ่งการตรวจสอบนั้นสามารถทำได้หลายรูปแบ
โดยเราต้องระบุถึง \textbf{ความผิดปรกติ และการล่อล่วง (Anomaly and fraud)} ก่อนเป็นอันดับแรก
ซึ่งการตรวจสอบนั้นสามารถทำได้ด้วยการหารูปแบบ และลักษณะของความผิดปรกติจากกรณีที่ทราบมาก่อนหน้านี้แล้ว
เช่น อาศัยการสร้างรูปแบบจากคุณลักษณะของเครือข่าย ทั้ง จำนวนผู้ที่โหนดนั้นมีปฏิสัมพันธ์ด้วย 
ลักษณะการเชื่อมต่อ จำนวนการเชื่อมต่อกับโหนดอื่น เป็นต้น เมื่อได้โมเดลข้างต้นแล้วก็สามารถที่จะนำมาตรวจสอบ
\textbf{พฤติกรรมก่อกวน (Spamming Behavior)} เช่น ใน Twitter ที่ตรวจสอบพบว่ามีผู้ใช้งานที่พยายามส่งข้อความเดิมๆ 
ซ้ำไปมาให้กับผู้ใช้งานรายอื่น จึงเข้าข่ายที่ระบุได้ว่าเป็นพฤติกรรมก่อกวน

\subsection{รูปแบบการนำเสนอ}

เนื่องจากลักษณะการใช้งาน{\OSN} นั้นมีได้อย่างหลากหลาย และรูปแบบความสัมพันธ์นั้นถูกกำหนดโดยผู้ใช้งาน
ดังนั้นวิธีการนำเสนอรูปแบบความสัมพันธ์(ที่หลากหลายนี้จึงได้ถูกคิดค้นขึ้นมาเพื่อทำให้สามาถสื่อสารข้อมูลได้ดียิ่งขึ้น
ทั้ง \textbf{รูปแบของโครงสร้าง (Structure model)} ที่พยายามสร้างขึ้นเพื่อให้สะท้อนคุณลักษณะของ{\SNS}
แต่ละรายที่มีลักษณะการใช้งานต่างกัน เช่น ของ Facebook ที่เน้นความสัมพันธ์คล้ายกับสังคมจริง, 
Flickr ที่เน้นเพียงแค่กระจายรูปภาพ โดยสามารถแบ่งได้เป็น \textbf{เชิงเดี่ยว (Sigleton)} 
ไม่มีความเชื่อมโยงใดๆ เลย เช่น Flickr, \textbf{กลุ่มเฉพาะ (Isolated Communities)}
ที่มีความสัมพันธ์เพียงแค่รอบๆ คนที่เป็นที่รู้จักเท่านั้น เช่น Path, และ \textbf{กลุ่มขนาดใหญ่ (Giant component)} 
ที่เชื่อมโยงผู้ใช้จำนวนมากเข้าด้วยกัน