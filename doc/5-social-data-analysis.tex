\section{\SDA}

เมื่อเกิดการใช้งานจากผู้ใช้งานที่หลากหลาย และแตกต่างกันในหลายบริบท 
ทำให้ข้อมูลที่เกิดขึ้นนั้นน่าสนใจเป็นอย่างมาก โดยเฉพาะอย่างยิ่ง 
ข้อมูลสำหรับนำมาประมวลผลเชิงภาษาธรรมชาติ (Natural Language Processing: NLP)
นอกจากนั้น การพัฒนาระบบที่สามารถรองรับกับข้อมูลในประมาณมากก็นับเป็นเรื่องที่ท้าทายเป็นอย่างยิ่ง

\subsection{การวิเคราะห์อารมณ์} 
ด้วยเหตุที่ว่าการใช้งาน{\OSN} ส่วนใหญ่เป็นการใช้งานผ่านตัวอักษร 
ที่ผู้ใช้งานมักจะถ่ายทอดอารมณ์และความคิดกับสิ่งที่ตนเองรู้สึกออกมาได้อย่างเสมอ
การเข้าใจว่าผู้ใช้งานนั้นรู้สึกเช่นไรด้วยการวิเคราะห์ข้อความนั้น 
จะช่วยให้สามารถนำข้อมูลเหล่านั้นไปกำหนดแนวทางการทำงานของผู้ที่เกี่ยวข้องได้ เช่น 
บริษัททราบอารมณ์ความรู้สึกต่อผลิตภัณฑ์ใหม่ที่พึ่งวางสู่ตลาดจากการวิเคราะห์ข้อความใน{\OSN} 
ซึ่งความท้าทายเป็นอย่างมากนั่นคือความเป็นอิสระในการป้อนข้อมูล เช่น 
ข้อความที่ได้นั้นอาจจะอยู่ในรูปของสแลง รูปภาพแทนอารมณ์ (emoji) 
หรือสัญลัษณ์บางอย่างที่ขึ้นอยู่กับเชื้อชาติและภาษาของผู้ใช้ เป็นต้น

\subsection{การทำนายผล}
จากที่ทราบว่า{\OSN}นั้นเปรียบได้เหมือนกับสภาพแวดล้มเสมือนจริง 
ที่สะท้อนกับสภาพสังคมด้านนอกเป็นอย่างดี เมื่อนำข้อมูลที่ได้มาวิเคราะห์แล้ว 
หัวข้อที่มักจะนำไปทำนายมักจะเป็นแนวโน้มของเรื่องที่สนใจ เช่น 
\textbf{
    การเลือกต้้ง, รายได้ของการฉายภาพยนต์, แนวโน้มของการขายหนังสือ, 
    การแพร่กระจายของโรคระบาด, 
    และการทำนายราคาหุ้นจากการวิเคราะห์อารมณ์ในเครือข่าย}

\subsection{การตรวจจับแนวโน้มความสนใจ}
อีกแนวทางการศึกษาวิจัยหนึ่งคือแนวโน้มของเรื่องที่เครือข่ายนั้นกำลังสนใจอยู่ 
หลายงานวิจัยที่พยายามสร้างตัวชี้วัดเพื่อตรวจจับสิ่งที่เครือข่ายนั้นกำลังสนใจอยู่ เช่น 
\textbf{การวิเคราะห์ข้อความ (Message analysis)} ด้วยการนำข้อความในเครือข่ายมาวิเคราะห์
ทั้งการหา \textit{tf-idf} ในการประมวลผลภาษาธรรมชาติ 
ที่จะทำให้เห็นความถี่ของข้อมูลที่แปลกประหลาดเกิดขึ้นในช่วงเวลาหนึ่งๆ 
หรือ \textbf{การตรวจจับเหตุการณ์ที่จริง (Detecting real events)} 
โดยอาศัยคุณลักษณะของเครือข่ายประกอบเข้าด้วยกัน เช่น พิกัดภูมิศาสตร์ ประกอบเข้าด้วยกัน
เพื่อทำนายกิจกรรมบางอย่างที่เกิดขึ้น เช่น คอนเสิร์ต การแจกรางวัน 
หรือแม้กระทั่งอุบัติเหตุ หรือภัยพิบัติธรรมชาติ

%%%%%%%%%%%%%%%%%%%%%%%%%%%%%%%%%%%%%%
\section{การวิเคราะห์ปฎิสัมพันธ์ในสังคม}
เมื่อมีผู้คนใช้งานเป็นจำนวนมากแล้ว 
ก็ย่อมหลีกเลี่ยงไม่ได้เลยที่จะเกิดการปฏิสัมพันธ์กันระหว่างผู้ใช้งานด้วยกัน
เกิดเป็นกลุ่มตามความสนใจ ลักษณะการจับกลุ่ม 
ซึ่งบางรูปแบบนั้นก็ยังเกิดเป็นคำถามสำหรับการวิจัยถึงพฤติกรรมเล่านี้ว่าจะสามารถเทียบได้กับพฤติกรรมในสังคมปรกติได้หรือไม่

\subsection{การแพร่กระจายข่ายลวง}
หัวข้อที่น่าสนใจเรื่องการมีปฏิสัมพันธ์นั่นส่วนหนึ่งคือการสร้างและแพร่กระจายข่าวลวง 
ซึ่งจำเป็นต้องอธิบาย \textbf{ลักษณะของข่าวลวง (Characterizing rumors)} 
จากผลการวิจัยนั้นพบว่ากระแสของข่าวลือมีมากกว่าข่าวโดยปรกติทั่วไปมาก 
ซึ่งมีลักษณะเช่นเดียวกันกับการศึกษาข้อมูลกับเครือข่ายปรกติทั่วไป ซึ่ง\textbf{การตรวจจับข่าวลวง (Detecting rumors)}
จากการศึกษาเครือข่ายข่าวลวงที่เกี่ยวกับภัยธรรมชาติที่จีนนั้น 
ได้พบกับรูปแบบของข่าวลวงที่กระจายจากแหล่งต่างๆ 
โดยการวิเคราะห์คุณลักษณะของข้อความที่โพสต์ ผู้ใช้งานที่มีส่วนร่วมในการสร้างข้อมูลดังกล่าว
และสุดท้ายที่น่าสนใจคือ \textbf{การจำกัดการแพร่กระจายของข่าวลวง (Rumor containment)} 
ที่มีการศึกษาโดยการค้นหาโทโลโลจีของเครือข่ายข่าวลวงนั้น และตัดการเชื่อมต่อโดยรอบนั้นออกไป
พบว่ามีประสิทธิภาพในการจำกัดข่าวลวงได้เป็นอย่างดี


%%%%%%%%%%%%%%%%%%%%%%%%%%%%%%%%%%%%%
\section{สรุป}
จากข้อมูลที่ได้สำรวจงานวิจัยมาข้างนั้นนั้น จะเห็นได้ว่า 
การศึกษา{\OSN}นั้นมีความหลากหลายมากกว่าเพียงแค่การวิเคราะห์ความสัมพันธ์ระหว่างโหนดเท่านั้น
เรายังสามารถใช้คุณลักษณะที่ผู้ให้{\SNS}จัดเตรียมไว้ให้เข้ามาประกอบด้วย 
เพื่อหาลักษณะพิเศษที่เกิดขึ้นภายในเครือข่าย ตามความสนใจของผู้วิจัย 
ดังนั้นหากศึกษาคุณลักษณะของเครือข่ายที่จะศึกษาก่อนแล้วกำหนดกรอบการวิจัยได้เป็นอย่างดีแล้ว 
จะทำให้ได้ผลลัพธ์การศึกษาออกมาตามทีต้องการได้
