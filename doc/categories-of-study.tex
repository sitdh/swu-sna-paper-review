\section{หัวข้อการศึกษา}

เนื่องจากแนวทางการศึกษาวิจัยใน{\OSN}นั้นมีหลากหลาย ดังนั้นเพื่อให้ง่ายต่อการเข้าใจ 
จึงได้แบ่งกลุ่มการศึกษาออกมาเป็น 3 กลุ่มด้วยกัน ได้แก่

\subsection{\STA} ถือเป็นกลุ่มงานแรกๆ ที่ต้องทำเมื่อศึกษา{\OSN} 
หลังจากดึงข้อมูลมาจากผู้ให้{\SNS}เรียบร้อยแล้วนั้น เพื่อให้เข้าใจคุณลักษณะต่างๆ 
ของข้อมูลที่ได้จากเครือข่าย การศึกษาวิจัยในกลุ่มนี้ยังคงได้รับความสนใจตั้งแต่เริ่มต้น 

\subsection{\SDA} เป็นแนวทางหนึ่งในการศึกษาวิจัย 
เพื่อให้ทราบถึงสิ่งที่เครือข่ายนี้สร้างขึ้นมา เช่น ลักษณะความสัมพันธ์ ความเชื่อมโย่ง 
หรือกลุ่มข้อมูลความสัมพันธ์กันภายในเครือข่าย

\subsection{\SIA} 
เมื่อเกิดความสัมพันธ์ขึ้นภายในเครือข่ายแล้วนั้น
สิ่งที่น่าสนใจต่อมากคือพฤติกรรมต่างๆ ของโหนด (ผู้ใช้งาน) 
ในเครื่องข่ายนั้นที่ต่างก็สร้างความสัมพันธ์ระหว่างโหนดในเครือข่าย 
ซึ่งการศึกษาในหัวข้อนี้นั้นจากข้อมูล พบว่าสามารถค้นพบข้อมูลได้หลายหลายหัวข้อ เช่น 
จิตวิทยา สังคมวิทยา หรือแม้กระทั่งชีววิทยา 