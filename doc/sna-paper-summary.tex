\documentclass[conference]{IEEEtran}
\IEEEoverridecommandlockouts
% The preceding line is only needed to identify funding in the first footnote. If that is unneeded, please comment it out.
\usepackage{cite}
\usepackage{amsmath,amssymb,amsfonts}
\usepackage{algorithmic}
\usepackage{graphicx}
\usepackage{textcomp}
\usepackage{xcolor}
\usepackage{fontspec}
\usepackage{xltxtra}
\usepackage{xunicode}
\usepackage{scrextend}

\XeTeXlinebreaklocale{'th'}
\newfontfamily\thaifont{TH Sarabun New}
\setmainfont{TH Sarabun New}
\changefontsizes[12pt]{12pt}
\XeTeXlinebreakskip = 0pt plus 1pt
\defaultfontfeatures{Scale=1.23}

\def\abstractname{บทคัดย่อ}
\def\IEEEkeywordsname{คำสำคัญ}
\def\OSN{เครือข่ายสังคมออนไลน์}
\def\SNS{บริการเครือข่ายสังคม}
\def\STA{การวิเคราะห์โครงสร้าง}
\def\SDA{การวิเคราะห์ข้อมูลสังคม}
\def\SIA{การวิเคราะห์ปฏิสัมพันธ์ทางสังคม}

\begin{document}
  \title{การวิเคราะห์{\OSN}: การสำรวจแนวทางการวิจัยในวิทยาการคอมพิวเตอร์}

  \author{\IEEEauthorblockN{สิทธิพงษ์ เหล่าโก้ก}
    \IEEEauthorblockA{
      ภาคการศึกษาที่ 2 ปีการศึกษา 2563 \\
      ภาควิชาวิทยาการคอมพิวเตอร์ คณะวิทยาศาสตร์ \\
      มหาวิทยาลัยศรีนครินทรวิโรฒ
    }
    \textit{sitdhibong.laokok@g.swu.ac.th}
  }

  \maketitle

  \begin{abstract}
    บทคัดย่อ
  \end{abstract}

  \begin{IEEEkeywords}
    {\OSN}, การสำรวจ, การวิจัยด้านคอมพิวเตอร์, การเรียนรู้ของเครื่อง, ระบบ 
  \end{IEEEkeywords}

  \section{บทนำ}
hello world

  \section{หัวข้อในการศึกษาเกี่ยวกับเครือข่ายสังคมออนไลน์}

\subsection{ทำไมเราต้องศึกษา{\OSN}}

  \section{หัวข้อการศึกษา}

เนื่องจากแนวทางการศึกษาวิจัยใน{\OSN}นั้นมีหลากหลาย ดังนั้นเพื่อให้ง่ายต่อการเข้าใจ 
จึงได้แบ่งกลุ่มการศึกษาออกมาเป็น 3 กลุ่มด้วยกัน ได้แก่

\subsection{\STA} ถือเป็นกลุ่มงานแรกๆ ที่ต้องทำเมื่อศึกษา{\OSN} 
หลังจากดึงข้อมูลมาจากผู้ให้{\SNS}เรียบร้อยแล้วนั้น เพื่อให้เข้าใจคุณลักษณะต่างๆ 
ของข้อมูลที่ได้จากเครือข่าย การศึกษาวิจัยในกลุ่มนี้ยังคงได้รับความสนใจตั้งแต่เริ่มต้น 

\subsection{\SDA} เป็นแนวทางหนึ่งในการศึกษาวิจัย 
เพื่อให้ทราบถึงสิ่งที่เครือข่ายนี้สร้างขึ้นมา เช่น ลักษณะความสัมพันธ์ ความเชื่อมโย่ง 
หรือกลุ่มข้อมูลความสัมพันธ์กันภายในเครือข่าย

\subsection{\SIA} 
เมื่อเกิดความสัมพันธ์ขึ้นภายในเครือข่ายแล้วนั้น
สิ่งที่น่าสนใจต่อมากคือพฤติกรรมต่างๆ ของโหนด (ผู้ใช้งาน) 
ในเครื่องข่ายนั้นที่ต่างก็สร้างความสัมพันธ์ระหว่างโหนดในเครือข่าย 
ซึ่งการศึกษาในหัวข้อนี้นั้นจากข้อมูล พบว่าสามารถค้นพบข้อมูลได้หลายหลายหัวข้อ เช่น 
จิตวิทยา สังคมวิทยา หรือแม้กระทั่งชีววิทยา 

  \section{\STA}

การเข้าใจโครงสร้าง เพื่อนำข้อมูลมาวิเคราะห์ได้นั้นเป็นใจความสำคัญของหัวข้อนี้
ทั้งนี้เพื่อให้เข้าใจว่าเครือข่ายถูกสร้างขึ้นได้อย่างไร เปรียบเทียบโครงสร้างกันในเครือข่าย 
หรือระหว่างผู้ให้{\SNS}เองก็ตาม หรืออาจสร้างโมเดลของการจับกลุ่มสังคมนี้ก็เป็นได้
ซึ่งการศึกษาที่ผ่านมา พบว่าในเครือข่ายเองมีความสัมพันธ์พิเศษบางอย่าง เช่น 
ระยะทางเฉลี่ยระหว่างโหนดนั้นลดลงเรื่อยๆ 
จำนวนการเชื่อมต่อของโหนดนั้นเป็นไปตามกฎกำลัง (Power law) กล่างคือ
การเชื่อมต่อจะมีมากในโหนดแรกๆ และลดลงไปอย่างรวดเร็วในโหนดที่ไกลออกไป 
ซึ่งภายในหัวข้อนี้มีประเด็นที่น่าสนใจดังนี้

\subsection{คุณลักษณะของโทโพโลจี}
การวิเคราะห์โทโพโลจีของเครือข่ายนั้นเปิดเผยคุณลักษณะที่น่าสนใจของกลุ่มสังคมได้เป็นอย่างดี 
ทั้งวิธีการจับกลุ่มในแนวทางที่ต่างกัน ยิ่งไปกว่านั้น การดึงความสัมพันธ์ออกมาจาก{\OSN} 
นั้นทำได้ง่ายกว่าเครือข่ายแบบออฟไลน์เป็นอย่างมาก ซึ่งในหัวข้อนี้จะมีประเด็นที่น่าสนใจได้แก่

\subsubsection{โครงสร้างของเครือข่ายเปิดเผยอะไรบ้าง} โดยส่วนใหญ่เรามักจะเปรียบเทียบคุณลักษณะ
และโครงสร้างของเครือข่ายใหม่ที่ศึกษา กับเครือข่ายเดิมที่มีข้อมูลอยู่เดิม 
หรือแม้กระทั่งโครงสร้างทางสังคมแบบออฟไลน์ก็ตาม อีกทั้งยังเทียบกับคุณลักษณะเดิมที่ทราบ 
เช่น การเชื่อมต่อระหว่างโหนดในเครือข่ายนี้เป็นไปตามกฎกำลังหรือไม่ หากลุ่มสังคม (Community) ในเครือข่าย
หรือสิ่งที่เครือข่ายนี้กำลังสนใจ เป็นต้น

\subsubsection{ในเครือข่ายหนึ่งประกอบด้วยหลายเครือข่าย} เรื่องที่น่าสนใจอย่างหนึ่งคือ
ภายในเครือข่ายนั้นเอง จะประกอบไปด้วยลักษณะของเครือข่ายย่อยอีกหลายแบบ เช่น 
เครือข่ายของความสนใจที่มีร่วมกันระหว่างโหนด หรือกลุ่มเพื่อน เป็นต้น 

\subsection{คุณลักษณะของการใช้งานและหน้าที่การใช้งาน}
ในบางกรณีนั้นนักวิจัยจะให้ความสนใจเกี่ยวกับความสามารถในการเข้าใจบริการนั้นๆ 
รวมถึงสิ่งที่จะได้ตอบรับกลับมาเมื่อวิเคราะห์การทำงานในเครือข่ายนี้ โดยที่ 

\subsubsection{รูปแบบของเครือข่าย} เป็นลักษณะเด่นที่ต้องให้ความสำคัญเป็นอย่างมาก
เพื่อให้ทราบถึงลักษณะการสร้างขึ้นของเครือข่ายย่อยๆ ภายในเครือข่ายที่สนใจนั้น  

\subsubsection{ลักษณะของผู้ใช้งาน} โดยทั่วไปแล้วจะแบ่งลักษณะผู้ใช้งานออกเป็น 3 ลักษณะด้วยกัน
นั่นคือ \textbf{ผู้กระจายสื่อ (Broadcasters)} ที่มันจะมีผู้ติดตามเป็นจำนวนมาก เช่น
บุคคลที่มีชื่อเสียง หรือเป็นที่รู้จัก \textbf{คนทั่วไป (Acquaintances)} 
เป็นผู้ใช้งานส่วนใหญ่ของระบบที่มักจะมีผู้ติดตามไม่แตกต่างจากจำนวนของคนที่ไปติดตามเท่าไหร่นัก
และสุดท้ายคือ \textbf{ผู้ประสงค์ร้าย (Miscreants)} ที่มีพฤติกรรมติดตามบุคคลอื่นเป็นจำนวนมาก
โดยไม่มีใครติดตามเลย

\subsection{การตรวจสอบสิ่งผิดปรกติและการล่อลวง}
หัวข้อนี้เป็นหัวข้อหนึ่งที่ท้าทายและมีความสำคัญ เนื่องจากเมื่อมีผู้ใช้งานมากยิ่งขึ้น 
โอกาสเกิดความผิดปรกติและการล่อล่วงนั้นย่อมมีมากขึ้นตามไปด้วย ซึ่งการตรวจสอบนั้นสามารถทำได้หลายรูปแบ
โดยเราต้องระบุถึง \textbf{ความผิดปรกติ และการล่อล่วง (Anomaly and fraud)} ก่อนเป็นอันดับแรก
ซึ่งการตรวจสอบนั้นสามารถทำได้ด้วยการหารูปแบบ และลักษณะของความผิดปรกติจากกรณีที่ทราบมาก่อนหน้านี้แล้ว
เช่น อาศัยการสร้างรูปแบบจากคุณลักษณะของเครือข่าย ทั้ง จำนวนผู้ที่โหนดนั้นมีปฏิสัมพันธ์ด้วย 
ลักษณะการเชื่อมต่อ จำนวนการเชื่อมต่อกับโหนดอื่น เป็นต้น เมื่อได้โมเดลข้างต้นแล้วก็สามารถที่จะนำมาตรวจสอบ
\textbf{พฤติกรรมก่อกวน (Spamming Behavior)} เช่น ใน Twitter ที่ตรวจสอบพบว่ามีผู้ใช้งานที่พยายามส่งข้อความเดิมๆ 
ซ้ำไปมาให้กับผู้ใช้งานรายอื่น จึงเข้าข่ายที่ระบุได้ว่าเป็นพฤติกรรมก่อกวน

\subsection{รูปแบบการนำเสนอ}

เนื่องจากลักษณะการใช้งาน{\OSN} นั้นมีได้อย่างหลากหลาย และรูปแบบความสัมพันธ์นั้นถูกกำหนดโดยผู้ใช้งาน
ดังนั้นวิธีการนำเสนอรูปแบบความสัมพันธ์(ที่หลากหลายนี้จึงได้ถูกคิดค้นขึ้นมาเพื่อทำให้สามาถสื่อสารข้อมูลได้ดียิ่งขึ้น
ทั้ง \textbf{รูปแบของโครงสร้าง (Structure model)} ที่พยายามสร้างขึ้นเพื่อให้สะท้อนคุณลักษณะของ{\SNS}
แต่ละรายที่มีลักษณะการใช้งานต่างกัน เช่น ของ Facebook ที่เน้นความสัมพันธ์คล้ายกับสังคมจริง, 
Flickr ที่เน้นเพียงแค่กระจายรูปภาพ โดยสามารถแบ่งได้เป็น \textbf{เชิงเดี่ยว (Sigleton)} 
ไม่มีความเชื่อมโยงใดๆ เลย เช่น Flickr, \textbf{กลุ่มเฉพาะ (Isolated Communities)}
ที่มีความสัมพันธ์เพียงแค่รอบๆ คนที่เป็นที่รู้จักเท่านั้น เช่น Path, และ \textbf{กลุ่มขนาดใหญ่ (Giant component)} 
ที่เชื่อมโยงผู้ใช้จำนวนมากเข้าด้วยกัน

  \section{\SDA}

เมื่อเกิดการใช้งานจากผู้ใช้งานที่หลากหลาย และแตกต่างกันในหลายบริบท 
ทำให้ข้อมูลที่เกิดขึ้นนั้นน่าสนใจเป็นอย่างมาก โดยเฉพาะอย่างยิ่ง 
ข้อมูลสำหรับนำมาประมวลผลเชิงภาษาธรรมชาติ (Natural Language Processing: NLP)
นอกจากนั้น การพัฒนาระบบที่สามารถรองรับกับข้อมูลในประมาณมากก็นับเป็นเรื่องที่ท้าทายเป็นอย่างยิ่ง

\subsection{การวิเคราะห์อารมณ์} 
ด้วยเหตุที่ว่าการใช้งาน{\OSN} ส่วนใหญ่เป็นการใช้งานผ่านตัวอักษร 
ที่ผู้ใช้งานมักจะถ่ายทอดอารมณ์และความคิดกับสิ่งที่ตนเองรู้สึกออกมาได้อย่างเสมอ
การเข้าใจว่าผู้ใช้งานนั้นรู้สึกเช่นไรด้วยการวิเคราะห์ข้อความนั้น 
จะช่วยให้สามารถนำข้อมูลเหล่านั้นไปกำหนดแนวทางการทำงานของผู้ที่เกี่ยวข้องได้ เช่น 
บริษัททราบอารมณ์ความรู้สึกต่อผลิตภัณฑ์ใหม่ที่พึ่งวางสู่ตลาดจากการวิเคราะห์ข้อความใน{\OSN} 
ซึ่งความท้าทายเป็นอย่างมากนั่นคือความเป็นอิสระในการป้อนข้อมูล เช่น 
ข้อความที่ได้นั้นอาจจะอยู่ในรูปของสแลง รูปภาพแทนอารมณ์ (emoji) 
หรือสัญลัษณ์บางอย่างที่ขึ้นอยู่กับเชื้อชาติและภาษาของผู้ใช้ เป็นต้น

\subsection{การทำนายผล}
จากที่ทราบว่า{\OSN}นั้นเปรียบได้เหมือนกับสภาพแวดล้มเสมือนจริง 
ที่สะท้อนกับสภาพสังคมด้านนอกเป็นอย่างดี เมื่อนำข้อมูลที่ได้มาวิเคราะห์แล้ว 
หัวข้อที่มักจะนำไปทำนายมักจะเป็นแนวโน้มของเรื่องที่สนใจ เช่น 
\textbf{
    การเลือกต้้ง, รายได้ของการฉายภาพยนต์, แนวโน้มของการขายหนังสือ, 
    การแพร่กระจายของโรคระบาด, 
    และการทำนายราคาหุ้นจากการวิเคราะห์อารมณ์ในเครือข่าย}

\subsection{การตรวจจับแนวโน้มความสนใจ}
อีกแนวทางการศึกษาวิจัยหนึ่งคือแนวโน้มของเรื่องที่เครือข่ายนั้นกำลังสนใจอยู่ 
หลายงานวิจัยที่พยายามสร้างตัวชี้วัดเพื่อตรวจจับสิ่งที่เครือข่ายนั้นกำลังสนใจอยู่ เช่น 
\textbf{การวิเคราะห์ข้อความ (Message analysis)} ด้วยการนำข้อความในเครือข่ายมาวิเคราะห์
ทั้งการหา \textit{tf-idf} ในการประมวลผลภาษาธรรมชาติ 
ที่จะทำให้เห็นความถี่ของข้อมูลที่แปลกประหลาดเกิดขึ้นในช่วงเวลาหนึ่งๆ 
หรือ \textbf{การตรวจจับเหตุการณ์ที่จริง (Detecting real events)} 
โดยอาศัยคุณลักษณะของเครือข่ายประกอบเข้าด้วยกัน เช่น พิกัดภูมิศาสตร์ ประกอบเข้าด้วยกัน
เพื่อทำนายกิจกรรมบางอย่างที่เกิดขึ้น เช่น คอนเสิร์ต การแจกรางวัน 
หรือแม้กระทั่งอุบัติเหตุ หรือภัยพิบัติธรรมชาติ

%%%%%%%%%%%%%%%%%%%%%%%%%%%%%%%%%%%%%%
\section{การวิเคราะห์ปฎิสัมพันธ์ในสังคม}
เมื่อมีผู้คนใช้งานเป็นจำนวนมากแล้ว 
ก็ย่อมหลีกเลี่ยงไม่ได้เลยที่จะเกิดการปฏิสัมพันธ์กันระหว่างผู้ใช้งานด้วยกัน
เกิดเป็นกลุ่มตามความสนใจ ลักษณะการจับกลุ่ม 
ซึ่งบางรูปแบบนั้นก็ยังเกิดเป็นคำถามสำหรับการวิจัยถึงพฤติกรรมเล่านี้ว่าจะสามารถเทียบได้กับพฤติกรรมในสังคมปรกติได้หรือไม่

\subsection{การแพร่กระจายข่ายลวง}
หัวข้อที่น่าสนใจเรื่องการมีปฏิสัมพันธ์นั่นส่วนหนึ่งคือการสร้างและแพร่กระจายข่าวลวง 
ซึ่งจำเป็นต้องอธิบาย \textbf{ลักษณะของข่าวลวง (Characterizing rumors)} 
จากผลการวิจัยนั้นพบว่ากระแสของข่าวลือมีมากกว่าข่าวโดยปรกติทั่วไปมาก 
ซึ่งมีลักษณะเช่นเดียวกันกับการศึกษาข้อมูลกับเครือข่ายปรกติทั่วไป ซึ่ง\textbf{การตรวจจับข่าวลวง (Detecting rumors)}
จากการศึกษาเครือข่ายข่าวลวงที่เกี่ยวกับภัยธรรมชาติที่จีนนั้น 
ได้พบกับรูปแบบของข่าวลวงที่กระจายจากแหล่งต่างๆ 
โดยการวิเคราะห์คุณลักษณะของข้อความที่โพสต์ ผู้ใช้งานที่มีส่วนร่วมในการสร้างข้อมูลดังกล่าว
และสุดท้ายที่น่าสนใจคือ \textbf{การจำกัดการแพร่กระจายของข่าวลวง (Rumor containment)} 
ที่มีการศึกษาโดยการค้นหาโทโลโลจีของเครือข่ายข่าวลวงนั้น และตัดการเชื่อมต่อโดยรอบนั้นออกไป
พบว่ามีประสิทธิภาพในการจำกัดข่าวลวงได้เป็นอย่างดี


%%%%%%%%%%%%%%%%%%%%%%%%%%%%%%%%%%%%%
\section{สรุป}
จากข้อมูลที่ได้สำรวจงานวิจัยมาข้างนั้นนั้น จะเห็นได้ว่า 
การศึกษา{\OSN}นั้นมีความหลากหลายมากกว่าเพียงแค่การวิเคราะห์ความสัมพันธ์ระหว่างโหนดเท่านั้น
เรายังสามารถใช้คุณลักษณะที่ผู้ให้{\SNS}จัดเตรียมไว้ให้เข้ามาประกอบด้วย 
เพื่อหาลักษณะพิเศษที่เกิดขึ้นภายในเครือข่าย ตามความสนใจของผู้วิจัย 
ดังนั้นหากศึกษาคุณลักษณะของเครือข่ายที่จะศึกษาก่อนแล้วกำหนดกรอบการวิจัยได้เป็นอย่างดีแล้ว 
จะทำให้ได้ผลลัพธ์การศึกษาออกมาตามทีต้องการได้


  \bibliography{KurkaGZ15}{}
  \bibliographystyle{plain}

\end{document}