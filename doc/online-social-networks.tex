\section{หัวข้อในการศึกษาเกี่ยวกับเครือข่ายสังคมออนไลน์}
ในหัวข้อนี้เป็นการอธิบายถึงความสำคัญของการศึกษา{\OSN} 
และทำไมการศึกษาในหัวข้อนี้ถึงได้สำคัญมากขึ้นเรื่อยๆ ข้อมูลที่น่าสนใจ 
รวมไปถึงเครื่องมือที่น่าสนใจที่มักจะใช้งานในการศึกษาวิจัยเพื่อวิเคราะห์ข้อมูล{\OSN}

\subsection{ทำไมเราต้องศึกษา{\OSN}}

เพื่อเข้าใจถึงปัญหาและความสำคัญของการศึกษานั้น ได้จัดกลุ่มของความน่าสนใจไว้ดังนี้
\begin{itemize}
    \item \textbf{ข้อมูลที่ปรากฎอยู่ในเครือข่าย} จากที่ได้กล่าวมาแล้วในข้างต้นว่า {\OSN} 
    นั้นมีแนวโน้มการใช้งานเพิ่มมากขึ้นเรื่อยๆ อย่างเช่น Twitter นั้นมีแนวโน้มมากขึ้นเรื่อยๆ 
    จากข้อมูลพบว่าในปี 2011 นั้นมีข้อมูลทวีตรายวันอยู่ที่ 50 ล้านทวีต และเพิ่มขึ้นมาเป็น 
    500 ล้านทวีตต่อวัน ในปี 2013 \cite{web:twitter-usange-statistics}
    ถือเป็นเรื่องถ้าท้ายส่วนหนึ่งในการจัดการข้อมูลที่มีขนาดใหญ่เช่นนี้
    และเหมาะแก่การศึกษาในหัวข้อ ข้อมูลมหัต (Big Data)

    \item \textbf{ผู้ใช้งานที่หลากหลาย} จากที่กล่าวว่าผู้ใช้งานที่หลากหลาย 
    ซึ่งเป็นลักษณะเฉพาะของ{\OSN} รวมถึงหลายผู้ให้บริการ ดังนั้นข้อมูลที่เข้ามาในระบบนั้นย่อมมีมากขึ้นเป็นเงาตามตัว
    ซึ่งด้วยลักษณะของผู้ใช้งานนั้นมีความหลากหลาย จึงส่งผลให้ข้อมูลนั้นย่อมหลากหลายตามไปด้วย
    ดังนั้น{\OSN} จึงถือเป็นแหล่งข้อมูลที่ดีมากสำหรับข้อความเพื่อใช้ในงานประมวลผลภาษาธรรมชาติ 
    (Natural Language Processing -- NLP)

    \item \textbf{การปฏิสัมพันธ์ในนามของผู้ใช้งาน} ผู้ใช้งานนั้นมีสิทธิใช้งานกำหนดลักษณะการใช้งาน{\OSN}ได้อย่างเต็มที่
    และสามารถเลือกได้ว่าจะสร้างความสัมพันธ์กับผู้ใช้รายอื่นอย่างไรบ้าง 
    ซึ่งรูปแบบความสัมพันธ์ที่เกิดขึ้นนั้นเป็นประเด็นที่น่าสนใจในการศึกษาวิจัยได้

    \item \textbf{การเปลี่ยนแปลงชั่วครั้งคราว} การใช้งาน{\OSN} นั้นผู้ใช้งานย่อมมีความสนใจที่หลากหลาย
    และเปลี่ยนไปได้เรื่อยๆ เสมอ ซึ่งการเปลี่ยนแปลงนี้เป็นเรื่องที่น่าสนใจในการวิจัย 
    เพื่อหาแนวโน้มของการเปลี่ยนแปลง รวมถึงหัวข้อที่น่าจะเกิดขึ้นต่อไป

    \item \textbf{รวดเร็ว} ใน{\OSN}นั้น ตอบสนองต่อเหตุการณ์อย่างรวดเร็วทั้งภายในและภายนอกระบบ

    \item \textbf{ใช้งานได้ทั่วไป} เมื่อเทคโนโลยีนั้นพัฒนามากขึ้นไปเรื่อยๆ 
    ส่งผลให้ผู้ใช้งานสามรถใช้งานได้เกือบทุกที่ทุกเวลา ส่งผลให้ข้อมูลนั้นสร้างได้ทุกเมื่อ 
    สิ่งที่ตามมานั่นคือข้อมูลเชิงภูมิศาสตร์จะปรากฎเข้ามาด้วยในเป็นอีกฟีเจอร์หนึ่งในข้อมูล{\OSN} 
    ซึ่งเป็นอีกแนวทางหนึ่งที่น่าสนใจเพื่อวิเคราะห์

\end{itemize}

\subsection{เครือขายที่ใช้สำหรับข้อมูล}
วิเคราะห์จาก 2 หัวข้อด้วยกัน นั่นคือ ปริมาณการใช้งานในเครือข่าย 
ว่ามีผู้ใช้งานมากน้อยขนาดไหน และความง่ายในการเข้าถึง ซึ่ง ณ 
เวลาที่ดำเนินการวิจัยนั้นพบว่า{\OSN}ที่มีขนาดใหญ่ที่สุดนั้นคือ \textit{Facebook} 
ด้วยจำนวนผู้ใช้งานกว่าพันล้านราย ทั้งนี้แบ่งกลุ่มเครื่อข่ายที่น่าสนใจดังนี้

\begin{itemize}
    \item \textit{Twitter} เป็นเครือข่ายที่น่าสนใจที่ใช้งานโดยการสื่อสารกันด้วยข้อความขนาดสั้นๆ 
    มากกว่านั้นยังเตรียมช่องทางการเชื่อมต่อให้กับนักพัฒนาไว้อย่างมาก ซึ่งข้อดีของครือข่ายนี้นอกจากการเข้าถึงได้ง่ายแล้ว
    ยังเป็นแหล่งข้อมูลที่หลากหลาย และเปลี่ยนแปลงอย่างรวดเร็วอีกด้วย
    
    \item \textit{Facebook} เครือข่ายขนาดใหญ่ที่เหมาะกับการวิเคราะห์ด้านความสัมพันธ์กันของบุคคลที่มีความหลากหลายทางด้าน 
    เชื้อชาติและวัฒนธรรม 

    \item \textit{YouTube} แหล่งรวมข้อมูลสื่อที่สามารถนำข้อมูลเหล่านั้นมาวิเคราะห์ถึงแนวโน้มความสนใจของผู้ใช้งาน 
    ในเวลาต่างๆ กันได้ 

    \item \textit{Flickr} เครือข่ายในลักษณะเดียวกันกับ YouTube ที่รวมเอาสื่อด้านรูปภาพไว้ด้วยกัน 
    สร้างกลุ่มความสัมพันธ์พิเศษของบุคคลที่มีทักษณะและความสนใจในแนวเดียวกันไว้
\end{itemize}

\subsection{เครื่องมือสำหรับช่วยวิเคราะห์}
ข้อมูลเครือข่ายใน{\OSN}นั้น นั้นมีความซับซ้อนมาก และการทำงานของ{\OSN}ส่วนใหญ่นั้น 
จะวิเคราะห์ข้อมูลทั้งหมดบนพื้นฐานของกราฟ (Graph-based) ซึ่งเครื่องมือที่มักจะใช้งานเพื่อวิเคราะห์ข้อมูลอยู่แล้วหลายรายการด้วยกัน
โดยแบ่งเป็นกลุ่มๆ ได้ดังนี้ 

\begin{itemize}
    \item \textbf{ฐานข้อมูลกราฟ (Graph database)}: สำหรับจัดเก็บข้อมูลของเครือข่ายที่จัดเก็บไว้เพื่อวิเคราะห์ข้อมูล 
    โดยฐานข้อมูลกราฟนี้จะมีคุณสมบัติพิเศษมากกว่าฐานข้อมูลเชิงสัมพันธ์ (Relational Database) ทั่วไป 
    เพราะสามารถจัดเก็บคุณสมบัติต่างๆ ของกราฟได้เป็นอย่างดี โดยมีตัวอย่างที่มักจะใช้งานกันอยู่นั่นคือ 
    AllegroGraph และ Neo4j

    \item \textbf{วาดกราฟ (Graph drawing)}: แสดงลักษณะเฉพาะพิเศษบางอย่างเพื่อวิเคราะห์เบื้องต้น
    หรือจัดเตรียมข้อมูลด้วยโครงสร้างข้อมูลสำหรับการแสดงผลกราฟ เช่น \textit{Graphviz, Tulip}

    \item \textbf{วิเคราะห์ (Analysis)}: เมื่อเครือข่ายมีขนาดใหญ่ขึ้น การค้นหาคุณลักษณะสำคัญบางอย่างนั้นย่อมทำได้ยากขึ้น 
    เรื่องมือที่สามารถช่วยวิเคราะห์เครือข่ายได้อย่างมีประสิทธิภาพนั้น \textit{NetworkX, igraph}

    \item \textbf{แสดงผล (Visualization)}: เพื่อการสื่อสารเกี่ยวกับโครงสร้างของกราฟนั้นชัดเจนมากยิ่งขึ้นเครื่องมือสำหรับวาดกราฟข้างต้นนั้น
    จะทำได้เพียงแค่รูปแบบพื้นฐานเพียงแค่ ขาว-ดำ เท่านั้น แต่เพื่อให้สามารถเข้าใจถึงโครงสร้างและคุณลักษณะพิเศษของกราฟมากยิ่งขึ้น
    จึงต้องใช้เครื่องมือเพิ่มเติมเพื่อกำหนดสี สัญลักษณ์แทนโหนด หรือข้อมูลอื่นๆ ประกอบ เพื่อทำให้การแสดงผลชัดเจนมากยิ่งขึ้นได้ ยกตัวอย่างเช่น \textit{Gephi, Cytoscape}
\end{itemize}