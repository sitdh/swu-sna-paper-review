\section{หัวข้อในการศึกษาเกี่ยวกับเครือข่ายสังคมออนไลน์}
ในหัวข้อนี้เป็นการอธิบายถึงความสำคัญของการศึกษา{\OSN} 
และทำไมการศึกษาในหัวข้อนี้ถึงได้สำคัญมากขึ้นเรื่อยๆ ข้อมูลที่น่าสนใจ 
รวมไปถึงเครื่องมือที่น่าสนใจที่มักจะใช้งานในการศึกษาวิจัยเพื่อวิเคราะห์ข้อมูล{\OSN}

\subsection{ทำไมเราต้องศึกษา{\OSN}}

เพื่อเข้าใจถึงปัญหาและความสำคัญของการศึกษานั้น ได้จัดกลุ่มของความน่าสนใจไว้ดังนี้
\begin{itemize}
    \item \textbf{ข้อมูลที่ปรากฎอยู่ในเครือข่าย} จากที่ได้กล่าวมาแล้วในข้างต้นว่า {\OSN} 
    นั้นมีแนวโน้มการใช้งานเพิ่มมากขึ้นเรื่อยๆ อย่างเช่น Twitter นั้นมีแนวโน้มมากขึ้นเรื่อยๆ 
    จากข้อมูลพบว่าในปี 2011 นั้นมีข้อมูลทวีตรายวันอยู่ที่ 50 ล้านทวีต และเพิ่มขึ้นมาเป็น 
    500 ล้านทวีตต่อวัน ในปี 2013 \cite{web:twitter-usange-statistics}
    ถือเป็นเรื่องถ้าท้ายส่วนหนึ่งในการจัดการข้อมูลที่มีขนาดใหญ่เช่นนี้
    และเหมาะแก่การศึกษาในหัวข้อ ข้อมูลมหัต (Big Data)

    \item \textbf{ผู้ใช้งานที่หลากหลาย} จากที่กล่าวว่าผู้ใช้งานที่หลากหลาย 
    ซึ่งเป็นลักษณะเฉพาะของ{\OSN} รวมถึงหลายผู้ให้บริการ ดังนั้นข้อมูลที่เข้ามาในระบบนั้นย่อมมีมากขึ้นเป็นเงาตามตัว
    ซึ่งด้วยลักษณะของผู้ใช้งานนั้นมีความหลากหลาย จึงส่งผลให้ข้อมูลนั้นย่อมหลากหลายตามไปด้วย
    ดังนั้น{\OSN} จึงถือเป็นแหล่งข้อมูลที่ดีมากสำหรับข้อความเพื่อใช้ในงานประมวลผลภาษาธรรมชาติ 
    (Natural Language Processing -- NLP)

    \item \textbf{การปฏิสัมพันธ์ในนามของผู้ใช้งาน} ผู้ใช้งานนั้นมีสิทธิใช้งานกำหนดลักษณะการใช้งาน{\OSN}ได้อย่างเต็มที่
    และสามารถเลือกได้ว่าจะสร้างความสัมพันธ์กับผู้ใช้รายอื่นอย่างไรบ้าง 
    ซึ่งรูปแบบความสัมพันธ์ที่เกิดขึ้นนั้นเป็นประเด็นที่น่าสนใจในการศึกษาวิจัยได้

    \item \textbf{การเปลี่ยนแปลงชั่วครั้งคราว} การใช้งาน{\OSN} นั้นผู้ใช้งานย่อมมีความสนใจที่หลากหลาย
    และเปลี่ยนไปได้เรื่อยๆ เสมอ ซึ่งการเปลี่ยนแปลงนี้เป็นเรื่องที่น่าสนใจในการวิจัย 
    เพื่อหาแนวโน้มของการเปลี่ยนแปลง รวมถึงหัวข้อที่น่าจะเกิดขึ้นต่อไป

    \item \textbf{รวดเร็ว} ใน{\OSN}นั้น ตอบสนองต่อเหตุการณ์อย่างรวดเร็วทั้งภายในและภายนอกระบบ

    \item \textbf{ใช้งานได้ทั่วไป} เมื่อเทคโนโลยีนั้นพัฒนามากขึ้นไปเรื่อยๆ 
    ส่งผลให้ผู้ใช้งานสามรถใช้งานได้เกือบทุกที่ทุกเวลา ส่งผลให้ข้อมูลนั้นสร้างได้ทุกเมื่อ 
    สิ่งที่ตามมานั่นคือข้อมูลเชิงภูมิศาสตร์จะปรากฎเข้ามาด้วยในเป็นอีกฟีเจอร์หนึ่งในข้อมูล{\OSN} 
    ซึ่งเป็นอีกแนวทางหนึ่งที่น่าสนใจเพื่อวิเคราะห์

\end{itemize}