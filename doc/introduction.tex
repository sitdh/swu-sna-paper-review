\section{บทนำ}
อินเทอร์เน็ตนั้นพัฒนาตัวเองมาขึ้นเรื่อยๆ มากกว่าที่เราคาดถึง 
ช่วยเชื่อมต่อเครื่องคอมพิวเตอร์ทั่วจากทุกมุมโลก ต่างเชื้อชาติ และวัฒนธรรมเข้าด้วยกัน 
มากไปกว่านั้นอินเทอร์เน็ตได้ช่วยสร้างกลุ่มสังคมประหนึ่งโลกเสมือนจริงอีกใบหนึ่งขึ้นมา 
ด้วยการเชื่อมโยงที่เรียกว่า{\OSN} ผ่านแอปพลิเคชันของผู้ให้บริการที่พัฒนาขึ้นมาโดยเฉพาะ
เช่น \textit{Twitter}, \textit{Google+}, และ \textit{Facebook} 
ที่มีผู้ใช้งานรวมกว่าหลายร้อยล้านคนเปรียบได้กับประเทศขนาดย่อยบนโลกออนไลน์ก็ว่าได้ 
ซึ่งจากการเก็บข้อมูลของผู้ให้บริการ พบว่าเวลาการใช้งานระบบเหล่านี้เพิ่มขึ้นเรื่อย 
เมื่อมีผู้ใช้งานที่หลากหลายทั้งด้วยพฤติกรรมการใช้งาน วัฒนธรรม การศึกษา 
รวมไปถึงคุณภาพการดำรงค์ชีวิตมาประกอบเข้าด้วยกันแล้วนั้น 
หากเราสามารถทำความเข้าใจบริบทที่หลากหลายเหล่านี้ได้ทั้งหมด 
ตลอดจนเข้าใจผลกระทบหรือการเปลี่ยนแปลงทั้งหมดเหล่านี้ได้ ย่อมเป็นเรื่องที่ท้าทายเป็นอย่างมาก 
ซึ่งในรายงานนี้สรุปความมาจากเอกสารวิชาการชื่อว่า \textit{Online Social Network Analysis: A Survey of Research Applications in Computer Science} \cite{DBLP:journals/corr/KurkaGZ15}
ได้สรุปรวมแนวทางที่น่าสนใจ และความท้าทายด้านต่างๆ ที่น่าสนใจสำหรับงานวิจัยรวมเข้าไว้ด้วยกัน 
พร้อมทั้งให้ตัวอย่างงานวิจัยที่สนใจประกอบด้วยด้วยกัน ดังนี้